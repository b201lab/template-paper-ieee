
\section{Desain dan Implementasi Sistem}
\label{sec:desainimplementasi}

\Cref{L1,L2} show listings typeset using the \texttt{lstlisting} environment.

\begin{lstlisting}[
  % one can adjust spacing here if required
  % aboveskip=2.5\baselineskip,
  % belowskip=-.8\baselineskip,
  caption={Example Java Listing},
  label=L1,
  language=Java,
  float]
public class Hello {
    public static void main (String[] args) {
        System.out.println("Hello World!");
    }
}
\end{lstlisting}

\begin{lstlisting}[
  % one can adjust spacing here if required
  % aboveskip=2.5\baselineskip,
  % belowskip=-.8\baselineskip,
  caption={Example XML Listing},
  label=L2,
  language=XML,
  float]
<example attr="demo">
  text content
</example>
\end{lstlisting}

\begin{figure}
  \includegraphics[width=.5\textwidth]{example-grid-100x100bp}
  \caption{Simple Figure. \cite[based on][]{mwe}}
  \label{fig:simple}
\end{figure}

\begin{figure*}
  \centering
  \includegraphics[width=.6\textwidth]{example-image-16x9}
  \caption{16x9 Figure}
  \label{fig:16x9}
\end{figure*}

\begin{figure*}[!b]
  \centering
  \subfloat[Case I]{\includegraphics[width=.4\textwidth]{example-image-a}%
    \label{fig_first_case}}
  \hfil
  \subfloat[Case II]{\includegraphics[width=.4\textwidth]{example-image-b}%
    \label{fig_second_case}}
  \caption{Simulation results for the network.}
  \label{fig_sim}
  % Note that often IEEE papers with subfigures do not employ subfigure
  % captions (using the optional argument to \subfloat[]), but instead will
  % reference/describe all of them (a), (b), etc., within the main caption.
  % Be aware that for subfig.sty to generate the (a), (b), etc., subfigure
  % labels, the optional argument to \subfloat must be present. If a
  % subcaption is not desired, just leave its contents blank,
  % e.g., \subfloat[].
\end{figure*}

\begin{table}
  \caption{Simple Table}
  \label{tab:simple}
  \centering
  \begin{tabular}{ll}
    \toprule
    Heading1 & Heading2 \\
    \midrule
    One      & Two      \\
    Thee     & Four     \\
    \bottomrule
  \end{tabular}
\end{table}

cref Demonstration: Cref at beginning of sentence, cref in all other cases.

\Cref{fig:simple} shows a simple fact, although \cref{fig:simple} could also show something else.
\Cref{fig:16x9} shows an 16x9 image spanning two columns.
\Cref{fig_first_case} is the first subfloat, whereas \Cref{fig_second_case} is the second one.

\Cref{tab:simple} shows a simple fact, although \cref{tab:simple} could also show something else.

\Cref{sec:latarbelakang} shows a simple fact, although \cref{sec:latarbelakang} could also show something else.

Brackets work as designed:
<test>
One can also input backquotes in verbatim text: \verb|`test`|.

The symbol for powerset is now correct: $\powerset$ and not a Weierstrass p ($\wp$).

\begin{inparaenum}
  \item All these items...
  \item ...appear in one line
  \item This is enabled by the paralist package.
\end{inparaenum}

``something in quotes'' using plain tex or use ``the enquote command''.

You can now write words containing hyphens which are hyphenated (application"=specific) at other places.
This is enabled by an additional configuration of the babel package.
In case you write ``application-specific'', then the word will only be hyphenated at the dash.

\lipsum[1-4]
