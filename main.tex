% Harus dimuat terlebih dahulu, digunakan agar file PDF memiliki format karakter yang benar.
% Untuk informasi lebih lanjut, lihat https://ctan.org/pkg/cmap.
\RequirePackage{cmap}

% Format dokumen sebagai paper konferensi menggunakan aturan IEEEtran terbaru (v1.8b).
% Untuk informasi lebih lanjut, lihat http://www.michaelshell.org/tex/ieeetran/.
\documentclass[conference]{IEEEtran}[2015/08/26]

% Format encoding font dan input menjadi 8-bit UTF-8.
\usepackage[T1]{fontenc}
\usepackage[utf8]{inputenc}

% Format bahasa menjadi bahasa german dan inggris.
\usepackage[indonesian]{babel}

% Digunakan untuk tujuan demonstrasi.
\usepackage{mwe}

% Digunakan untuk menampilkan font dengan style yang lebih baik.
\usepackage[zerostyle=b,scaled=.75]{newtxtt}

% Digunakan untuk menampilkan tabel dengan style yang lebih baik.
\usepackage{booktabs}

% Digunakan untuk menampilkan gambar pada dokumen.
\usepackage{graphicx}

% Digunakan untuk menampilkan potongan kode.
\usepackage{listings}
\lstset{
  basicstyle=\ttfamily,
  columns=fixed,
  basewidth=.5em,
  xleftmargin=0.5cm,
  captionpos=b
}

% Digunakan agar backticks (`) dapat dirender pada PDF.
% Untuk informasi lebih lanjut, lihat https://tex.stackexchange.com/a/341057/9075.
\usepackage{upquote}

% Digunakan untuk menyeimbangkan bagian akhir dokumen dengan dua kolom.
\usepackage{balance}

% Digunakan untuk menampilkan pustaka.
\usepackage[square,comma,numbers,sort&compress]{natbib}

% Mengubah format ukuran teks pada natbib.
\renewcommand{\bibfont}{\normalfont\footnotesize}

% Menambah nama penulis ketika menggunakan perintah \citet.
% Untuk informasi lebih lanjut, lihat https://tex.stackexchange.com/a/76075/9075.
\usepackage{etoolbox}
\makeatletter
\patchcmd{\NAT@test}{\else \NAT@nm}{\else \NAT@hyper@{\NAT@nm}}{}{}
\makeatother

% Digunakan untuk melakukan linewrap pada pustaka dengan url yang panjang
% jika terdapat hyphens
\usepackage[hyphens]{url}

% Digunakan untuk menambah hyperlink pada referensi.
\usepackage{hyperref}

% Menonaktifkan warna dan bookmark pada hyperref.
\hypersetup{hidelinks,
  colorlinks=true,
  allcolors=black,
  pdfstartview=Fit,
  breaklinks=true
}

% Digunakan untuk membenarkan hyperref pada gambar.
\usepackage[all]{hypcap}

% Digunakan untuk menampilkan beberapa gambar
\usepackage[caption=false,font=footnotesize]{subfig}

\usepackage{stfloats}

% Tambahkan format tanda hubung yang benar di sini
\hyphenation{
  ro-ket
  me-ngem-bang-kan
  per-hi-tu-ngan
}

\begin{document}

  % Ubah kalimat berikut sesuai dengan judul penelitian.
\title{Kalkulasi Energi pada Roket Luar Angkasa \\ Berbasis \emph{Anti-Gravitasi}}

% Ubah kalimat-kalimat berikut sesuai dengan nama, institusi, alamat dan kontak penulis.
\author{
  \IEEEauthorblockN{Elon Reeve Musk}
  \IEEEauthorblockA{Departemen Teknik Dirgantara\\
    Fakultas Teknologi Dirgantara\\
    Institut Teknologi Sepuluh Nopember\\
    Surabaya, Indonesia 60111\\
    elon.musk@mhs.its.ac.id}

  \and
  \IEEEauthorblockN{Nikola Tesla}
  \IEEEauthorblockA{Departemen Teknik Dirgantara\\
    Fakultas Teknologi Dirgantara\\
    Institut Teknologi Sepuluh Nopember\\
    Surabaya, Indonesia 60111\\
    \url{https://nikolatesla.me}}

  \and
  \IEEEauthorblockN{Wernher von Braun}
  \IEEEauthorblockA{Departemen Teknik Dirgantara\\
    Fakultas Teknologi Dirgantara\\
    Institut Teknologi Sepuluh Nopember\\
    Surabaya, Indonesia 60111\\
    von.braun@td.its.ac.id}
}

% Digunakan untuk menampilkan judul dan deskripsi penulis.
\maketitle

  % Mengubah keterangan `Abstract` ke bahasa indonesia.
% Hapus bagian ini untuk mengembalikan ke format awal.
\renewcommand\abstractname{Abstrak}

\begin{abstract}

  % Ubah paragraf berikut sesuai dengan abstrak dari penelitian.
  Pada penelitian ini kami mengajukan \lipsum[1][1-12]

\end{abstract}

% Mengubah keterangan `Index terms` ke bahasa indonesia.
% Hapus bagian ini untuk mengembalikan ke format awal.
\renewcommand\IEEEkeywordsname{Kata kunci}

\begin{IEEEkeywords}

  % Ubah kata-kata berikut sesuai dengan kata kunci dari penelitian.
  Roket, Anti-gravitasi, Energi, Angkasa.

\end{IEEEkeywords}


  % Ubah bagian berikut sesuai dengan konten-konten yang akan dimasukkan pada dokumen
  % Ubah judul dan label berikut sesuai dengan yang diinginkan.
\section{Pendahuluan}
\label{sec:pendahuluan}

% Ubah paragraf-paragraf pada bagian ini sesuai dengan yang diinginkan.

Pesatnya perkembangan roket yang merupakan \lipsum[2-4]

Pembahasan pada paper ini dimulai dengan presentasi mengenai penelitian lain (Bagian \ref{sec:penelitianterkait}).
Kemudian dilanjutkan dengan penjelasan mengenai arsitektur dari sistem yang dibuat (Bagian \ref{sec:arsitektur}).
Berdasarkan hal tersebut, kami menunjukkan lorem ipsum (Bagian \ref{sec:loremipsum}).
Terakhir, didapatkan kesimpulan dari penelitian yang telah dilakukan (Bagian \ref{sec:kesimpulan}).

  % Ubah judul dan label berikut sesuai dengan yang diinginkan.
\section{Penelitian Terkait}
\label{sec:penelitianterkait}

% Ubah paragraf-paragraf pada bagian ini sesuai dengan yang diinginkan.

Beberapa penelitian lain pernah dilakukan seperti yang dirumuskan oleh \citet{newton1687} bahwa \lipsum[5]
Hasil tersebut kemudian menjadi persamaan \ref{eq:hukumpertama}.

% Contoh pembuatan persamaan ilmiah.
\begin{equation}
  \label{eq:hukumpertama}
  \sum \mathbf{F} = 0\; \Leftrightarrow\; \frac{\mathrm{d} \mathbf{v} }{\mathrm{d}t} = 0.
\end{equation}

\lipsum[6-7]

  % Ubah judul dan label berikut sesuai dengan yang diinginkan.
\section{Arsitektur}
\label{sec:arsitektur}

% Ubah paragraf-paragraf pada bagian ini sesuai dengan yang diinginkan.

\subsection{Cetak Biru Roket}
\label{subsec:cetakbiruroket}

Pada cetak biru yang tertera pada Gambar \ref{fig:cetakbiru}. \lipsum[8]

% Contoh input gambar pada kolom.
\begin{figure} [ht]
  \centering
  % Ubah sesuai dengan nama file gambar dan ukuran yang akan digunakan.
  \includegraphics[width=0.4\textwidth]{gambar/cetakbiru.jpg}

  % Ubah sesuai dengan keterangan gambar yang diinginkan.
  \caption{Cetak biru roket yang akan diuji coba. \cite{cetakbiruspacex}}
  \label{fig:cetakbiru}
\end{figure}

\lipsum[9-10]

\subsection{Lorem Ipsum}
\label{subsec:loremipsum}

\lipsum[11]

% Contoh pembuatan tabel.
\begin{table}
  \caption{Contoh tabel sederhana}
  \label{tab:tabelsederhana}
  \centering
  \begin{tabular}{lll}
    \toprule
    Heading1 & Heading2 & Heading3  \\
    \midrule
    One      & Two      & Three     \\
    Four     & Five     & Six       \\
    \bottomrule
  \end{tabular}
\end{table}

% Contoh pembuatan potongan kode.
\begin{lstlisting}[
  language=C++,
  caption={Program halo dunia.},
  label={lst:halodunia}
]
#include <iostream>

int main() {
    std::cout << "Halo Dunia!";
    return 0;
}
\end{lstlisting}

\lipsum[12]

% Contoh pembuatan daftar.
\begin{enumerate}
  \item \lipsum[13][1-4]
  \item \lipsum[13][5-8]
  \item \lipsum[13][9-12]
\end{enumerate}

\lipsum[14-15]

  % Ubah judul dan label berikut sesuai dengan yang diinginkan.
\section{Lorem ipsum}
\label{sec:loremipsum}

% Ubah paragraf-paragraf pada bagian ini sesuai dengan yang diinginkan.

% Contoh input beberapa gambar pada halaman.
\begin{figure*}
  \centering
  \subfloat[Hasil A]{\includegraphics[width=.4\textwidth]{example-image-a}
    \label{fig:hasila}}
  \hfil
  \subfloat[Hasil B]{\includegraphics[width=.4\textwidth]{example-image-b}
    \label{fig:hasilb}}
  \caption{Contoh input beberapa gambar.}
  \label{fig:hasil}
\end{figure*}

\lipsum[16-18]

% Contoh input potongan kode dari file.
\lstinputlisting[
  language=Python,
  caption={Program perhitungan bilangan prima.},
  label={lst:bilanganprima}
]{program/bilangan-prima.py}

\lipsum[19-20]

  \section{Kesimpulan}
\label{sec:kesimpulan}

\blindtext


  % Menampilkan daftar pustaka dengan format IEEE
  \bibliographystyle{IEEEtranN}
  \bibliography{pustaka/pustaka.bib}

  % Menyeimbangkan bagian akhir di kedua kolom
  \balance

\end{document}
