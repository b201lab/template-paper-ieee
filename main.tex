% Harus dimuat terlebih dahulu, digunakan agar file PDF memiliki format karakter yang benar.
% Untuk informasi lebih lanjut, lihat https://ctan.org/pkg/cmap.
\RequirePackage{cmap}

% Format dokumen sebagai paper konferensi menggunakan aturan IEEEtran terbaru (v1.8b).
% Untuk informasi lebih lanjut, lihat http://www.michaelshell.org/tex/ieeetran/
\documentclass[conference]{IEEEtran}[2015/08/26]

% Digunakan untuk menyeimbangkan bagian akhir dokumen dengan dua kolom.
\usepackage{balance}

% Digunakan untuk menampilkan font dengan style yang lebih baik.
\usepackage[zerostyle=b,scaled=.75]{newtxtt}

% Digunakan untuk tujuan demonstrasi.
\usepackage{mwe}

% Format encoding font dan input menjadi 8-bit UTF-8.
\usepackage[T1]{fontenc}
\usepackage[utf8]{inputenc}

% Digunakan untuk memasukkan gambar pada dokumen.
\usepackage{graphicx}

% Format bahasa menjadi bahasa german dan inggris.
\usepackage[indonesian]{babel}

% Digunakan agar backticks (`) dapat dirender pada PDF.
% Untuk informasi lebih lanjut, lihat https://tex.stackexchange.com/a/341057/9075.
\usepackage{upquote}

% Digunakan untuk membuat list di dalam paragraf.
\usepackage{paralist}

% Digunakan untuk memperbaiki tipografi seperti font kerning.
% Untuk informasi lebih lanjut, lihat https://ctan.org/pkg/microtype.
\RequirePackage[final,expansion=alltext,protrusion=alltext-nott]{microtype}

%tweak \url{...}
\usepackage{url}
%\urlstyle{same}
%improve wrapping of URLs - hint by http://tex.stackexchange.com/a/10419/9075
\makeatletter
\g@addto@macro{\UrlBreaks}{\UrlOrds}
\makeatother
%nicer // - solution by http://tex.stackexchange.com/a/98470/9075
%DO NOT ACTIVATE -> prevents line breaks
%\makeatletter
%\def\Url@twoslashes{\mathchar`\/\@ifnextchar/{\kern-.2em}{}}
%\g@addto@macro\UrlSpecials{\do\/{\Url@twoslashes}}
%\makeatother

% Diagonal lines in a table - http://tex.stackexchange.com/questions/17745/diagonal-lines-in-table-cell
% Slashbox is not available in texlive (due to licensing) and also gives bad results. This, we use diagbox
%\usepackage{diagbox}

\usepackage{booktabs}

% Required for package pdfcomment later
\usepackage{xcolor}

% For listings
\usepackage{listings}
\lstset{%
  basicstyle=\ttfamily,%
  columns=fixed,%
  basewidth=.5em,%
  xleftmargin=0.5cm,%
  captionpos=b}%

% Enable nice comments
\usepackage{pdfcomment}
%
\newcommand{\commentontext}[2]{\colorbox{yellow!60}{#1}\pdfcomment[color={0.234 0.867 0.211},hoffset=-6pt,voffset=10pt,opacity=0.5]{#2}}
\newcommand{\commentatside}[1]{\pdfcomment[color={0.045 0.278 0.643},icon=Note]{#1}}
%
% Compatibility with packages todo, easy-todo, todonotes
\newcommand{\todo}[1]{\commentatside{#1}}
% Compatiblity with package fixmetodonotes
\newcommand{\TODO}[1]{\commentatside{#1}}

% Bibliopgraphy enhancements
%  - enable \cite[prenote][]{ref}
%  - enable \cite{ref1,ref2}
% Alternative: \usepackage{cite}, which enables \cite{ref1, ref2} only (otherwise: Error message: "White space in argument")
%
% Doc: http://texdoc.net/natbib
\ifCLASSOPTIONcompsoc
  % IEEE Computer Society needs nocompress option at cite.sty
  % natbib includes the same functionality
  \usepackage[%
    square,        % for square brackets
    comma,         % use commas as separators
    numbers,       % for numerical citations;
    sort           % orders multiple citations into the sequence in which they appear in the list of references;
    %sort&compress % as sort but in addition multiple numerical citations
                   % are compressed if possible (as 3-6, 15);
  ]{natbib}
\else
  % normal IEEE
  \usepackage[%
    square,        % for square brackets
    comma,         % use commas as separators
    numbers,       % for numerical citations;
    %sort           % orders multiple citations into the sequence in which they appear in the list of references;
    sort&compress % as sort but in addition multiple numerical citations
                   % are compressed if possible (as 3-6, 15);
  ]{natbib}
\fi
% Same fontsize as without natbib
\renewcommand{\bibfont}{\normalfont\footnotesize}
% Enable hyperlinked author names in the case of \citet
% Source: https://tex.stackexchange.com/a/76075/9075
\usepackage{etoolbox}
\makeatletter
\patchcmd{\NAT@test}{\else \NAT@nm}{\else \NAT@hyper@{\NAT@nm}}{}{}
\makeatother

% Enable that parameters of \cref{}, \ref{}, \cite{}, ... are linked so that a reader can click on the number an jump to the target in the document
\usepackage{hyperref}
% Enable hyperref without colors and without bookmarks
\hypersetup{hidelinks,
  colorlinks=true,
  allcolors=black,
  pdfstartview=Fit,
  breaklinks=true}
%
% Enable correct jumping to figures when referencing
\usepackage[all]{hypcap}

%enable \cref{...} and \Cref{...} instead of \ref: Type of reference included in the link
\usepackage[capitalise,nameinlink]{cleveref}
\crefname{lstlisting}{\lstlistingname}{\lstlistingname}
\Crefname{lstlisting}{Listing}{Listings}

%Following definitions are outside of IfPackageLoaded; inside, they are not visible
%
%Intermediate solution for hyperlinked refs. See https://tex.stackexchange.com/q/132420/9075 for more information.
\newcommand{\Vlabel}[1]{\label[line]{#1}\hypertarget{#1}{}}
\newcommand{\lref}[1]{\hyperlink{#1}{\FancyVerbLineautorefname~\ref*{#1}}}

\newenvironment{listing}[1][htbp!]{\begin{figure}[#1]}{\end{figure}}
\newcounter{listing}

\usepackage{xspace}
%\newcommand{\eg}{e.\,g.\xspace}
%\newcommand{\ie}{i.\,e.\xspace}
\newcommand{\eg}{e.\,g.,\ }
\newcommand{\ie}{i.\,e.,\ }

%introduce \powerset - hint by http://matheplanet.com/matheplanet/nuke/html/viewtopic.php?topic=136492&post_id=997377
\DeclareFontFamily{U}{MnSymbolC}{}
\DeclareSymbolFont{MnSyC}{U}{MnSymbolC}{m}{n}
\DeclareFontShape{U}{MnSymbolC}{m}{n}{
  <-6>    MnSymbolC5
  <6-7>   MnSymbolC6
  <7-8>   MnSymbolC7
  <8-9>   MnSymbolC8
  <9-10>  MnSymbolC9
  <10-12> MnSymbolC10
  <12->   MnSymbolC12%
}{}
\DeclareMathSymbol{\powerset}{\mathord}{MnSyC}{180}

% *** SUBFIGURE PACKAGES ***
\ifCLASSOPTIONcompsoc
  \usepackage[caption=false,font=footnotesize,labelfont=sf,textfont=sf]{subfig}
\else
  \usepackage[caption=false,font=footnotesize]{subfig}
\fi

\usepackage{stfloats}

% correct bad hyphenation here
\hyphenation{op-tical net-works semi-conduc-tor}

\begin{document}
%\IEEEoverridecommandlockouts

% Ubah kalimat berikut sesuai dengan judul penelitian.
\title{Kalkulasi Energi pada Roket Luar Angkasa \\ Berbasis \emph{Anti-Gravitasi}}

% Ubah kalimat-kalimat berikut sesuai dengan nama, institusi, alamat dan kontak penulis.
\author{
  \IEEEauthorblockN{Elon Reeve Musk}
  \IEEEauthorblockA{Departemen Teknik Dirgantara\\
    Fakultas Teknologi Dirgantara\\
    Institut Teknologi Sepuluh Nopember\\
    Surabaya, Indonesia 60111\\
    elon.musk@mhs.its.ac.id}

  \and
  \IEEEauthorblockN{Nikola Tesla}
  \IEEEauthorblockA{Departemen Teknik Dirgantara\\
    Fakultas Teknologi Dirgantara\\
    Institut Teknologi Sepuluh Nopember\\
    Surabaya, Indonesia 60111\\
    \url{https://nikolatesla.me}}

  \and
  \IEEEauthorblockN{Wernher von Braun}
  \IEEEauthorblockA{Departemen Teknik Dirgantara\\
    Fakultas Teknologi Dirgantara\\
    Institut Teknologi Sepuluh Nopember\\
    Surabaya, Indonesia 60111\\
    von.braun@td.its.ac.id}
}

% Digunakan untuk menampilkan judul dan deskripsi penulis.
\maketitle


% In case you want to add a copyright statement.
%
% Source: https://tex.stackexchange.com/a/200330/9075
%
% All possible solutions:
%  - https://tex.stackexchange.com/a/325013/9075
%  - https://tex.stackexchange.com/a/279134/9075
%  - https://tex.stackexchange.com/q/279789/9075 (TikZ)
%  - https://tex.stackexchange.com/a/200330/9075 - for non-compsocc papers
\iffalse
    \makeatletter
    \def\ps@IEEEtitlepagestyle{%
      \def\@oddfoot{\mycopyrightnotice}%
      \def\@evenfoot{}%
    }
    \makeatother
    \def\mycopyrightnotice{%
      \begin{minipage}{\textwidth}
        \footnotesize
        1551-3203 \copyright 2015 IEEE.
        Personal use is permitted, but republication/redistribution requires IEEE permission.
        \\
        See \url{https://www.ieee.org/publications_standards/publications/rights/index.html} for more information.
      \end{minipage}
      \gdef\mycopyrightnotice{}% just in case
    }
\fi

% Mengubah keterangan `Abstract` ke bahasa indonesia.
% Hapus bagian ini untuk mengembalikan ke format awal.
\renewcommand\abstractname{Abstrak}

\begin{abstract}

  % Ubah paragraf berikut sesuai dengan abstrak dari penelitian.
  Pada penelitian ini kami mengajukan \lipsum[1][1-12]

\end{abstract}

% Mengubah keterangan `Index terms` ke bahasa indonesia.
% Hapus bagian ini untuk mengembalikan ke format awal.
\renewcommand\IEEEkeywordsname{Kata kunci}

\begin{IEEEkeywords}

  % Ubah kata-kata berikut sesuai dengan kata kunci dari penelitian.
  Roket, Anti-gravitasi, Energi, Angkasa.

\end{IEEEkeywords}


% For peer review papers, you can put extra information on the cover
% page as needed:
% \ifCLASSOPTIONpeerreview
% \begin{center} \bfseries EDICS Category: 3-BBND \end{center}
% \fi
%
% For peerreview papers, this IEEEtran command inserts a page break and
% creates the second title. It will be ignored for other modes.
\IEEEpeerreviewmaketitle

% Ubah judul dan label pada bagian ini sesuai dengan yang diinginkan
\section{Latar Belakang}
\label{sec:latarbelakang}

% Ubah paragraf-paragraf pada bagian ini sesuai dengan yang diinginkan

Pesatnya perkembangan roket yang merupakan \lipsum[2-4]

Pembahasan pada paper ini dimulai dengan presentasi mengenai penelitian lain (\cref{sec:penelitianterkait}).
Kemudian pembahasan dilanjutkan dengan penjelasan mengenai desain dan implementasi dari sistem yang dibuat (\cref{sec:desainimplementasi}).
Berdasarkan hal tersebut, kami menunjukkan lorem ipsum (\cref{sec:loremipsum}).
Terakhir, didapatkan kesimpulan dari penelitian yang telah dilakukan (\cref{sec:kesimpulan}).

% Ubah judul dan label berikut sesuai dengan yang diinginkan.
\section{Penelitian Terkait}
\label{sec:penelitianterkait}

% Ubah paragraf-paragraf pada bagian ini sesuai dengan yang diinginkan.

Beberapa penelitian lain pernah dilakukan seperti yang dirumuskan oleh \citet{newton1687} bahwa \lipsum[5]
Hasil tersebut kemudian menjadi persamaan \ref{eq:hukumpertama}.

% Contoh pembuatan persamaan ilmiah.
\begin{equation}
  \label{eq:hukumpertama}
  \sum \mathbf{F} = 0\; \Leftrightarrow\; \frac{\mathrm{d} \mathbf{v} }{\mathrm{d}t} = 0.
\end{equation}

\lipsum[6-7]


\section{Desain dan Implementasi Sistem}
\label{sec:desainimplementasi}

\Cref{L1,L2} show listings typeset using the \texttt{lstlisting} environment.

\begin{lstlisting}[
  % one can adjust spacing here if required
  % aboveskip=2.5\baselineskip,
  % belowskip=-.8\baselineskip,
  caption={Example Java Listing},
  label=L1,
  language=Java,
  float]
public class Hello {
    public static void main (String[] args) {
        System.out.println("Hello World!");
    }
}
\end{lstlisting}

\begin{lstlisting}[
  % one can adjust spacing here if required
  % aboveskip=2.5\baselineskip,
  % belowskip=-.8\baselineskip,
  caption={Example XML Listing},
  label=L2,
  language=XML,
  float]
<example attr="demo">
  text content
</example>
\end{lstlisting}

\begin{figure}
  \includegraphics[width=.5\textwidth]{example-grid-100x100bp}
  \caption{Simple Figure. \cite[based on][]{mwe}}
  \label{fig:simple}
\end{figure}

\begin{figure*}
  \centering
  \includegraphics[width=.6\textwidth]{example-image-16x9}
  \caption{16x9 Figure}
  \label{fig:16x9}
\end{figure*}

\begin{figure*}[!b]
  \centering
  \subfloat[Case I]{\includegraphics[width=.4\textwidth]{example-image-a}%
    \label{fig_first_case}}
  \hfil
  \subfloat[Case II]{\includegraphics[width=.4\textwidth]{example-image-b}%
    \label{fig_second_case}}
  \caption{Simulation results for the network.}
  \label{fig_sim}
  % Note that often IEEE papers with subfigures do not employ subfigure
  % captions (using the optional argument to \subfloat[]), but instead will
  % reference/describe all of them (a), (b), etc., within the main caption.
  % Be aware that for subfig.sty to generate the (a), (b), etc., subfigure
  % labels, the optional argument to \subfloat must be present. If a
  % subcaption is not desired, just leave its contents blank,
  % e.g., \subfloat[].
\end{figure*}

\begin{table}
  \caption{Simple Table}
  \label{tab:simple}
  \centering
  \begin{tabular}{ll}
    \toprule
    Heading1 & Heading2 \\
    \midrule
    One      & Two      \\
    Thee     & Four     \\
    \bottomrule
  \end{tabular}
\end{table}

cref Demonstration: Cref at beginning of sentence, cref in all other cases.

\Cref{fig:simple} shows a simple fact, although \cref{fig:simple} could also show something else.
\Cref{fig:16x9} shows an 16x9 image spanning two columns.
\Cref{fig_first_case} is the first subfloat, whereas \Cref{fig_second_case} is the second one.

\Cref{tab:simple} shows a simple fact, although \cref{tab:simple} could also show something else.

\Cref{sec:latarbelakang} shows a simple fact, although \cref{sec:latarbelakang} could also show something else.

Brackets work as designed:
<test>
One can also input backquotes in verbatim text: \verb|`test`|.

The symbol for powerset is now correct: $\powerset$ and not a Weierstrass p ($\wp$).

\begin{inparaenum}
  \item All these items...
  \item ...appear in one line
  \item This is enabled by the paralist package.
\end{inparaenum}

``something in quotes'' using plain tex or use ``the enquote command''.

You can now write words containing hyphens which are hyphenated (application"=specific) at other places.
This is enabled by an additional configuration of the babel package.
In case you write ``application-specific'', then the word will only be hyphenated at the dash.

\lipsum[1-4]

% Ubah judul dan label berikut sesuai dengan yang diinginkan.
\section{Lorem ipsum}
\label{sec:loremipsum}

% Ubah paragraf-paragraf pada bagian ini sesuai dengan yang diinginkan.

% Contoh input beberapa gambar pada halaman.
\begin{figure*}
  \centering
  \subfloat[Hasil A]{\includegraphics[width=.4\textwidth]{example-image-a}
    \label{fig:hasila}}
  \hfil
  \subfloat[Hasil B]{\includegraphics[width=.4\textwidth]{example-image-b}
    \label{fig:hasilb}}
  \caption{Contoh input beberapa gambar.}
  \label{fig:hasil}
\end{figure*}

\lipsum[16-18]

% Contoh input potongan kode dari file.
\lstinputlisting[
  language=Python,
  caption={Program perhitungan bilangan prima.},
  label={lst:bilanganprima}
]{program/bilangan-prima.py}

\lipsum[19-20]

\section{Kesimpulan}
\label{sec:kesimpulan}

\blindtext


% trigger a \newpage just before the given reference
% number - used to balance the columns on the last page
% adjust value as needed - may need to be readjusted if
% the document is modified later
%\IEEEtriggeratref{8}
% The "triggered" command can be changed if desired:
%\IEEEtriggercmd{\enlargethispage{-5in}}

% Enable to reduce spacing between bibitems (source: https://tex.stackexchange.com/a/25774)
% \def\IEEEbibitemsep{0pt plus .5pt}

\bibliographystyle{IEEEtranN} % IEEEtranN is the natbib compatible bst file
% argument is your BibTeX string definitions and bibliography database(s)
\bibliography{pustaka/pustaka.bib}

\balance

\end{document}
